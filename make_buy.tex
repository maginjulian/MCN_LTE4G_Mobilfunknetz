
%----------------------------------------------------------------
%
%  File    :  make_buy.tex
%
%  Authors :  ???
% 
%  Created :  7 Sept 2019
% 
%  Changed :  7 Sept 2019
% 
%---------------------------------------------------------------

\section{Make or Buy}
\label{sec:make_buy}

Kosten und Preise wurden nicht ermittelt. Die erwähnten Anbieter haben kaum Preise öffentlich ausgeschrieben, welche vergleichbar sind. Kosten variieren, je nach Umfang und tatsächlich eingesetzter Hardware, stark. Grundsätzlich ist jeglicher aktueller Service erwerbbar und hängt von dem Budget ab, welches ausgegeben wird. Die Kosten der Open-Source Variante liegen in der Umsetzung und dem Testen. Der Vorteil besteht darin, dass diese an beliebige Bedürfnisse angepasst werden kann. In diesem Fall ist es notwendig, verglichen mit proprietären Anbietern, mehr Geld für Schulungen, für den Aufbau von Know-How und Wartungsmaßnahmen aufzuwenden. Der Aufbau des Know-Hows kann zudem genutzt werden, um ein proprietäres Produkt inklusive Dienstleistungen zu erzeugen. So ist es möglich, diese später zu verwenden, um bei der Anbindung anderer Inseln mittels LTE zu unterstützen und die initialen Kosten zu senken. Auch die Errichtung eines Ausbildungszentrums ist denkbar. Ein Nachteil dieser Lösung ist, dass ein Sicherheitsfaktor besteht. Im Problemfall kann kein externer Anbieter zu Wartungszwecken hinzugezogen werden, welcher dieses System wartet und das Problem behebt. Dies ist Teil der Verträge mit proprietären Anbietern.

\subsection{Entscheidungsbegründung}
Für den Fall der Anbindung anderer Systeme, oder dem Zusammenführen von weiteren Systemen, bieten die Open-Source Lösungen bereits eine gute Unterstützung. Ausweitungen auf weitere Inseln kann daher, ohne Festlegung auf einen speziellen Anbieter, durchgeführt werden. Ein funktionsfähiges LTE Netzwerk wird, sofern dies von einem proprietären Anbieter umgesetzt wird, schneller zur Verfügung stehen. Grund hierfür ist, dass es keinen Einarbeitungsaufwand gibt und die Umsetzung von Fachpersonal durchgeführt wird, wodurch auch die Kosten steigen. Des Weiteren ist das Risiko des Scheiterns, bei einem völlig selbst entwickelten Netzwerks, höher. Trotz der aufgeführten Nachteile, ist eine Umsetzung mittels einer Open-Source Lösung zu präferieren. Diese ist preisgünstig umzusetzen. Ein Umstieg auf einen etablierten Anbieter kann auch später erfolgen, sollte es zu unvorhergesehen Problemen kommen.

%\subsection{Kriterien für Make or Buy Entscheidung}


%\subsubsection{Differenzierung der Antennenleistung}
%\subsubsection{Testbarkeit}


%\subsubsection{Begründung}