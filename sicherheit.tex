
%----------------------------------------------------------------
%
%  File    :  make_buy.tex
%
%  Authors :  ???
% 
%  Created :  7 Sept 2019
% 
%  Changed :  7 Sept 2019
% 
%---------------------------------------------------------------

\section{Sicherheit}
\label{sec:sicherheit}
\subsection{Kritische Infrastrukturen}
Um die Inselbewohner vor Wirtschaftsspionage, Diebstahl und Manipulation zu schützen, ist die Thematik der Sicherheit von Bedeutung. Da Cyber-Kriminelle heutzutage sehr professionell agieren, um ihr Schadensausmaß möglichst maximal zu gestalten, sind die kritischen Infrastrukturen der Inseleinrichtungen wie zB die Energieversorgung und Kommunikationssysteme von Spitälern besonders zu schützen. Um zur politischen und sozialen Stabilität auf der Insel beizutragen, sind somit Versorgungsengpässe beziehungsweise das Risiko von Ausfällen in kritischen Infrastrukturen zu reduzieren. Denn Beeinträchtigungen in diesen Bereichen können zu kaskadierenden Effekten in anderen Lebensbereichen führen. Die Digitalisierung der Insel erfordert besondere Schutzmaßnahmen, sodass durch Prävention, Detektion und Reaktion für sowohl dem Staat, der heimischen Wirtschaft und eben auch der Gesellschaft trotz zunehmenden Abhängigkeiten von Informationstechnologien und -Vernetzung auch in Zukunft zuverlässig kritische Infrastrukturen funktionieren\cite{BSI17}.

Laut Bundesamt für Sicherheit in der Informationstechnik (BSI) sind potentielle Gefährdungen im Zusammenhang mit öffentlichen Mobilfunknetzen bei der Planung und der Errichtung vor allem auch bei der Wartung ebendieser nicht außer Acht zu lassen. Ebenso sollten mögliche Gegenmaßnahmen in Erwägung gezogen werden, um den Schutz vertraulicher Daten erhöhen zu können \cite{Ger08}.
\subsection{Sicherheitsgefährdungen und Schutzmaßnahmen}
Erfolgt ein Zugriff auf die implementierten technischen Einrichtungen kann dies zu Missbrauch durch Unbefugte führen. Dies kann durch Schwachstellen bei der Endgeräte-Authentisierung, bei potenziellen Unsicherheiten in der Datenverschlüsselung, durch eine unzureichende Verschlüsselungsstärke, die Erstellung von Bewegungsprofilen durch Ortungsmaßnahmen, die Unterbindung von Mobilfunkkommunikation durch Störsender oder durch Software-Manipulation der UE geschehen.\ Um hier nur auf einige von vielen Sicherheitsgefährdungen wie zB Over-The-Air Programming (OTA), Proxy-Manipulation, FOTA – Firmware Over The Air oder eine mangelnde Implementierung von Sicherheitsmechanismen  einzugehen, ist folgendes zu sagen: Mögliche Schutzmaßnahmen dienen vor allem zur Prüfung von Software-Abhängigkeiten oder dem Einfordern von Sicherheitszertifizierungen. 
Betrachtet man Übertragungsverfahren, Datenübertragungsraten sowie weitere Technologien im Bereich von Mobilfunknetzen, ist zu verzeichnen, dass zwar die Sicherheitsrisiken wesentlich komplexer wurden, gleichzeitig paradoxerweise das Bewusstsein der Nutzer allerdings nicht in gleichem Ausmaß angestiegen ist. Aus diesem Grund kann es durchaus nützlich sein, dieses Bewusstsein dezidierter zu schärfen\cite{Ger08}.

\subsection{Regelungen für Telekommunikationsunternehmen}

Da sich unter den Inselbewohnern auch IT-versierte Juristen befinden, wurde ein IT-Sicherheitsgesetz erlassen, welches die Telekommunikationsunternehmen verpflichtet, die IT Infrastrukturen nach dem Stand der Technik angemessen abzusichern. In einem Zyklus von Minimum alle 24 Monate muss die Sicherheit geprüft werden.
Des weiteren besteht eine Verpflichtung der Anbieter gegenüber den Inselbewohnern, diese zu warnen, sofern es Grund zur Annahme gibt, dass die UEs für IT-Angriffe missbraucht wurden. Zusätzlich herrscht eine Informationspflicht dahingehend, dass der User auf möglichst viele Arten hingewiesen wird, wie etwaige Störungen zu entfernen sind.
Natürlich sollen IT-Sicherheitsmaßnahmen nicht nur zum Schutz von personenbezogenen Daten der Inselbewohner, sondern auch zum Schutz vor unerlaubten Eingriffen in die Infrastruktur selbst eingesetzt werden.
Auch zu beachten ist, das auftretende IT-Sicherheitsvorfälle bei den zuständigen Behörden gemeldet werden.\cite{BSI17}  