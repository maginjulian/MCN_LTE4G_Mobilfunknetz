
%----------------------------------------------------------------
%
%  File    :  make_buy.tex
%
%  Authors :  ???
% 
%  Created :  7 Sept 2019
% 
%  Changed :  7 Sept 2019
% 
%---------------------------------------------------------------

\section{Sicherheit}
\label{sec:sicherheit}
\subsection{Schutz kritischer Infrastrukturen}

Julian.
Um die Inselbewohner vor Wirtschaftsspionage, Diebstahl und Manipulation zu schützen ist die Thematik der Sicherheit konsequenterweise nicht zu vernachlässigen. Da Cyber-Kriminelle heutzutage sehr professionell agieren, um ihr Schadensausmaß möglichst maximal zu gestalten, sind die kritischen Infrastrukturen (KI) der Inseleinrichtungen wie z.B. die Energieversorgung und Kommunikationssysteme von Spitälern besonders zu schützen. Um zur politischen und sozialen Stabilität auf der Insel beizutragen sind somit Versorgungsengpässe beziehungsweise das Risiko von Ausfällen von KI zu reduzieren. Denn Beeinträchtigungen in diesen Bereichen können zu kaskadierenden Effekten auf andere Lebensbereiche führen. Die Digitalisierung der Insel erfordert besondere Schutzmaßnahmen, sodass durch Prävention, Detektion und Reaktion für sowohl dem Staat, der heimischen Wirtschaft und ebenauch der Gesellschaft trotz zunehmender IT-Abhängigkeit und -Vernetzung auch in Zukunft zuverlässig KIs funktioniere.\cite{BSI17} 

Laut Bundesamt für Sicherheit in der Informationstechnik (BSI) sind potentielle Gefährdungen im Zusammenhang mit öffentlichen Mobilfunknetzen bei der Planung und Errichtung vor allem Wartung ebendieser nicht außer Acht zu lassen. Ebenso sollten mögliche Gegenmaßnahmen in Erwägung gezogen werden, um den Schutz konfidenzieller Daten erhöhen zu können. \cite{Ger08}

\subsection{Regelungen für Telekommunikationsunternehmen}

Da sich unter den Inselbewohnern auch IT-versierte Juristen befinden, wurde ein IT-Sicherheitsgesetz erlassen, welches die Telekommunikationsunternehmen verpflichtet, die IT Infrastrukturen nach dem Stand der Technik angemessen abzusichern. In einem Zyklus von Minimum alle 24 Monate muss die Sicherheit geprüft werden.
Des weiteren besteht eine Verpflichtung der Anbieter gegenüber den Inselbewohnern, diese zu warnen, sofern es Grund zur Annahme gibt, das die UE für IT-Angriffe missbraucht wurden. Zusätzlich herrscht eine Informationspflicht dahingehend, dass auf möglichst viele Arten hingewiesen wird, wie etwaige Störungen zu entfernen sind.
Natürlich sollen IT-Sicherheitsmaßnahmen nicht nur zum Schutz von personenbezogenen Inselbewohnerdaten, sondern auch zum Schutz vor unerlaubten Eingriffen in die Infrastruktur eingesetzt werden.
Auch zu beachten ist, das auftretende IT-Sicherheitsvorfälle gemeldet werden bei den zuständigen Behörden.\cite{BSI17}  