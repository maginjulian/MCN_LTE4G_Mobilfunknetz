
%----------------------------------------------------------------
%
%  File    :  conclusion.tex
%
%  Authors :  ???
% 
%  Created :  7 Sept 2019
% 
%  Changed :  7 Sept 2019
% 
%---------------------------------------------------------------

\section{Conclusion}
\label{sec:conclusion}
\subsection{Zusammenfassung der Ergebnisse}
\label{subsec:zusammenfassung}
In der Arbeit wird ein Überblick über LTE gegeben. Des Weiteren wird sowohl auf den Aufbau als auch die einzelnen Komponenten näher eingegangen. Ein interessanter Aspekt ist neben den diversen Anbietern von LTE Netzen der Aufbau eines eigenen Netzes mittels Open-Source Software. Da die benötigte Hardware kostengünstig zu erwerben ist, die Software flexibel in Bezug auf Anpassungen und Erweiterungen ist, wäre der Aufbau und Betrieb eines LTE Netzes hiermit vorzuziehen.\ % \ref{tab:Übersicht Use Cases und Softwarelösungen}
Tabelle I fasst die einsetzbaren Technologien zusammen und hebt ihre Eignung je Use Case hervor.\ Neben dem Erkenntnis- und Wissensgewinn eines solchen Projektes, ist der Preis deutlich geringer, verglichen mit proprietären Anbietern. Sollte es zu unvorhergesehenen Problemen kommen, gibt es immer noch die Möglichkeit auf einen Lieferanten eines Netzwerks zu wechseln. Zusätzliches Know-How wird, bei Bedarf, auch bei den Entwicklern der Open-Source Lösung angefragt. 

Eine Ausstattung einer Insel mittels LTE, ist zudem für einen Open-Source Anbieter ein Prestigeprojekt. OpenAirInterface ist die zu bevorzugende Lösung. Diese ist erprobt und bietet eine vollständige Dokumentation an. Unter den Mitgliedern und Partnern der OpenAirInterfaceTM Software Alliance finden sich auch Firmen und Institutionen wie RedHat, Nokia, Fujitsu, das Frauenhofer Institut sowie zahlreiche Universitäten (Technische Universität München, University of Utha, University of Sydney uvm). Kooperationen mit einem oder mehreren Mitgliedern und Partnern ist nicht ausgeschlossen. Vielmehr ist dies für die erfolgreiche Umsetzung anzustreben.
Das erworbene Wissen sollte zudem für weitere Projekte im Open-Source Bereich genutzt werden. Sowohl von den Mitarbeitern der Insel selbst, wie auch von beteiligten Partnern. Da bei dieser Art der Umsetzung Wissen erworben wird, ist von einem späteren Zeitpunkt des vollen LTE Betriebes auszugehen, verglichen mit dem Zukauf sämtlicher Hard- und Softwarekomponenten und Dienstleistungen bei den erwähnten Anbietern.

\subsection{Ausblick}
\label{subsec:ausblick}
Da Mobilfunknetze evolutionären Weiterentwicklungsprozessen unterliegen, wird es in Zukunft interessant sein, welche weiteren Entwicklungen zu erwarten sind. Hierzu sind zB New Spectrum mm-Waves, Smart Cells, Beamforming, Massive MIMO, Network Slicing, Mobile Edge Computing, Network Function Virtualization bzw Software Defined Networks  für Bereiche wie Enhanced Mobile Broadband, Ultra Reliable Low Latency Communication oder Massive Machine Type Communication zu zählen%\cite{Wol19}
. Wird das 4G Netz der Insel in ein 5G Netz umgebaut, ist das E-UTRAN entsprechend dem R15 schon auf dem 5G Standard.