
%----------------------------------------------------------------
%
%  File    :  einleitung.tex
%
%  Authors :  Eisenhut
% 
%  Created :  7 Sept 2019
% 
%  Changed :  9 Sept 2019
% 
%---------------------------------------------------------------

\section{Einleitung}
\label{sec:einleitung}
% The very first letter is a 2 line initial drop letter followed
% by the rest of the first word in caps.
\IEEEPARstart{W}{elche} Infrastruktur ist für ein State-of-the-art  4.~ Generation/Long-Term Evolution (4G/LTE) \abk{4G/LTE}{4. Generation/Long-Term Evolution} Mobilfunknetz unter dem Gesichtspunkt der Kosteneffizienz auf einer Insel mit ungefähr 500 Einwohnern geeignet, ist die zentrale Frage dieser Arbeit.
% You must have at least 2 lines in the paragraph with the drop letter
% (should never be an issue)
\subsection{Gliederung}
\label{subsec:gliederung}
Nach der Einleitung mit der \nameref{subsec:gliederung} und den \nameref{subsec:praemissen} folgt das Kapitel~\nameref{sec:aufbau}, welches die grundsätzliche Netzarchitektur erklärt. Danach werden im Kapitel~\nameref{sec:Komponenten} die benötigten Bestandteile vorgestellt. Das Kapitel \nameref{sec:open_source} stellt alternative Open-Source Lösungen vor. Das Kapitel \nameref{sec:make_buy} stellt die vorgeschlagene Lösung vor und im Kapitel \nameref{sec:sicherheit} werden Risiken aufgezeigt, bevor die Arbeit schließlich mit der \nameref{sec:conclusion} endet.
\subsection{Prämissen}
\label{subsec:praemissen}
Es wird von einer Insel in der Südsee mit 500 Einwohnern ausgegangen. In dieser Arbeit wird nur der Telefon und Datendienst des Netzes untersucht. Die Anbindung an internationale Netze ist nicht Inhalt der Arbeit. Die Staatsform der Insel ist die parlamentarische Republik. Deren gewählte Vertreter treffen die Entscheidung über das zu erbauende Mobilfunknetz. Um das Zusammenleben der Einwohner zu regeln, tritt auf der Insel eine Rechtsprechung in Kraft, welche die Einhaltung von legislativen Vorgängen überwacht. Somit unterliegen auf der Insel die Telekommunikationsanbieter auch gesetzlichen Regelungen und Vorgaben.


\subsection{Bewertungskriterien}
\label{subsec: Bewertungskriterien}
Aufgrund der Tatsache, dass sich das 4G LTE Mobilfunknetz über verschiedene Sektoren erstrecken soll, ergeben sich verschiedene technische und nicht-technische Anforderungen an eine solche Architektur, je nach spezifiziertem Anwendungsfall. Zu diesen gehören der Öffentliche Dienst, die Energieversorgung, das Transportwesen, das Krisen-und Katastrophenmanagement, wie auch der Bildungsbereich, oder generell zu Entertainment Zwecken wie für die Nutzung von Streaming Plattformen, sowie auch das Gesundheitswesen.\ Hierbei kristallisieren sich insbesondere Kriterien heraus, welche einerseits die Zellreichweite, eine hohe Abdeckung, sowie andererseits hohe Datenraten berücksichtigen. Diese Metriken sind insofern von Bedeutung, damit die Inselbewohner letztlich als Benutzer des Mobilfunknetzes die angebotenen Services und Dienstleistungen mit hoher Zuverlässigkeit und entsprechend geringen Ausfallwahrscheinlichkeiten nutzen können.\cite{Tch18}. 

	\subsection{Use Case Seenotrettungssystem}
\label{subsec:Use Cases}


%es wäre interessant noch zu überlegen welche konkreten Anforderungen diese haben (z.b. sind autonome Autos wohl vermutlich sehr latenzsensitiv und erfordern hohe reliability, die Seenotrettung benötigt vermutlich eher eine hohe Reichweite/Coverage auf See) 

Die zu betrachtende Insel verfügt über ein Seenotrettungssystem, präziser gesagt über ein maritimes Rettungs-Koordinationszentrum, über welches mit Hilfe von Rettungsschiffen und Rettungsdrohnen in Seenot geratene Inselbewohner lokalisiert werden können%\cite{eckermann2018tinylte}
.\ Im Hinblick auf ein Seenotrettungssystem liegt die Vermutung nahe, dass eine hohe Reichweite, respektive auch Coverage auf hoher See anzustreben und wünschenswert ist.\ Zu bedenken in diesem Zusammenhang ist, dass die Coverage des Systems stark von der Verfügbarkeit stationärer Basisstationen abhängig ist.\ Sollte die Coverage nicht ausreichen, gibt es die Möglichkeit, sich Coverage Extension Anwendungen zu bedienen.\ 

\subsection{Use Case autonome vehikulare Mobilität}
\label{subsec: Autonome Vehikel}
Des Weiteren bewegen sich die Einwohner, auf dem Staatsgebiet der Insel, in autonom fahrenden Autos, weshalb auch hierfür entsprechende Anforderungen in Bezug auf die Komponenten für vehikulare Kommunikation bei der Planung eines Mobilfunknetzes berücksichtigt werden sollten. Der Einsatz von selbstfahrenden Automobilen führt auch zu Überlegungen in Bezug auf mögliche Verzögerungen bei der Übertragung von Latenz sensitiven Daten, die für ein funktionierendes autonomes Verkehrssystem  essentiell sind. Somit ist es notwendig, eine hohe Dienstleistungsqualität auch bekannt als Quality of Service (QoS) zu garantieren. Die Details zur Umsetzung von QoS sind in Abschnitt \nameref{subsec:logkomponenten} näher erläutert.\ Dies geschieht vor allem unter dem Aspekt von Safety Bedenken, um Kollisionen bei der Navigation zu verhindern.\ 

Bisherige Experimente von Eckermann et al.\ mit stationären \nameref{subsec:tinyLTE} Knoten demonstrieren eine mögliche Reichweite von einem Zellradius bis zu 175~m bei einer Fahrgeschwindigkeit von ca. 1.4~ m/s (5~km/h), siehe dazu auch\ Abschnitt \nameref{subsec:tinyLTE}\ zur näheren Beschreibung. Dabei war der mobile Knoten an einem Fahrzeug auf einer Höhe von 1,65~m montiert. Durch das Aufzeichnen der GPS Signale konnte eine Map erstellt werden, die sowohl Signalstärke als auch Latenzen zusammenfasst. Bei einer Entfernung von 110~m wurde die Übertragung sehr unzuverlässig. Alles in allem wurde eine Latenz von 7~ms festgehalten.\ Bezogen auf den Standard 3GPP V2X ist eine maximale Latenz von 100~ms für Safety bezogene V2X Kommunikation und 20~ms für drohende Unfälle mit einer Empfangszuverlässigkeit von mehr als 80 bis zu 95 Prozent zulässig.\ In Erwägung gezogen wird auch die Skalierbarkeit der beteiligten Kommunikationsknoten.\ \cite{eckermann2018tinylte} Da eine Infrastruktur von Grund auf erbaut wird, muss keine Rücksicht auf etwaige bestehende Legacy Architekturen genommen werden, in welche neue Komponenten zu integrieren sind. 

	\subsection{Anbieter am Markt}
\label{subsec:Anbieter am Markt}
Eine Vielzahl von Anbietern für die Errichtung eines Mobilfunknetzes bestimmen derzeit den globalen LTE Advanced Markt.\ Die dominierenden Unternehmen in diesem sind Ambra Solutions, Arris International,
Athonet, Cisco, Comba, DruidSoftware, Ericsson, Future Technologies, General Dynamics und 
Huawei. Neben diesen existieren aber auch weitere Anbieter wie beispielsweise Lemko, Luminate Wireless,
Mavenir,
NEC,
Netnumber,
Nokia,
Pdvwireless,
Quortus,
Redline Communications,
Samsung,
Sierra Wireless,
Star Solutions,
Ursys,
Verizon und 
Zinwave.\cite{Max19}
Abhängig von den Use Cases ergeben sich bei der Analyse und beim Design spezifizierte funktionale wie auch nicht-funktionale Anforderungen. Durch die konkrete Betrachtung einer Einführung eines Mobilfunknetzes gilt es, sich mit auftretenden Fragestellungen wie dem Angebot an qualifiziertem Fachpersonal, welches auf den LTE Advanced Standard geschult ist, auseinanderzusetzen. Dieses kann dann sowohl in den Phasen der Analyse und des Designs als auch in denen der Implementierung sowie bei der Wartung und beim Testen vor Inbetriebnahme gezielt eingesetzt werden. So ist beispielsweise zu prüfen, ob entsprechendes Kapital vorhanden ist, um externe Berater zur Projektunterstützung einzufliegen. Sofern diese schon einmal ähnliche Projekte realisiert haben, ist es sinnvoll mit Hilfe dieser Erfahrungswerte von vergangenen Projekten in Bezug auf Planung und Koordination Aufwandsabschätzungen vorzunehmen.\ Somit lassen sich Fehleinschätzungen minimieren, im Idealfall gar vermeiden. Sind generell Fachkräfte mit Domänen spezifischem Know-How verfügbar dann ist es sinnvoll Schulungszentren aufzubauen, um langfristig Abhängigkeiten von Schulungsdienstleistern zu verringern.



%\subsubsection{Subsubsection Heading Here}

% An example of a floating figure using the graphicx package.
% Note that \label must occur AFTER (or within) \caption.
% For figures, \caption should occur after the \includegraphics.
% Note that IEEEtran v1.7 and later has special internal code that
% is designed to preserve the operation of \label within \caption
% even when the captionsoff option is in effect. However, because
% of issues like this, it may be the safest practice to put all your
% \label just after \caption rather than within \caption{}.
%
% Reminder: the "draftcls" or "draftclsnofoot", not "draft", class
% option should be used if it is desired that the figures are to be
% displayed while in draft mode.
%
%\begin{figure}[!t]
%\centering
%\includegraphics[width=2.5in]{myfigure}
% where an .eps filename suffix will be assumed under latex, 
% and a .pdf suffix will be assumed for pdflatex; or what has been declared
% via \DeclareGraphicsExtensions.
%\caption{Simulation results for the network.}
%\label{fig_sim}
%\end{figure}

% Note that the IEEE typically puts floats only at the top, even when this
% results in a large percentage of a column being occupied by floats.


% An example of a double column floating figure using two subfigures.
% (The subfig.sty package must be loaded for this to work.)
% The subfigure \label commands are set within each subfloat command,
% and the \label for the overall figure must come after \caption.
% \hfil is used as a separator to get equal spacing.
% Watch out that the combined width of all the subfigures on a 
% line do not exceed the text width or a line break will occur.
%
%\begin{figure*}[!t]
%\centering
%\subfloat[Case I]{\includegraphics[width=2.5in]{box}%
%\label{fig_first_case}}
%\hfil
%\subfloat[Case II]{\includegraphics[width=2.5in]{box}%
%\label{fig_second_case}}
%\caption{Simulation results for the network.}
%\label{fig_sim}
%\end{figure*}
%
% Note that often IEEE papers with subfigures do not employ subfigure
% captions (using the optional argument to \subfloat[]), but instead will
% reference/describe all of them (a), (b), etc., within the main caption.
% Be aware that for subfig.sty to generate the (a), (b), etc., subfigure
% labels, the optional argument to \subfloat must be present. If a
% subcaption is not desired, just leave its contents blank,
% e.g., \subfloat[].


% An example of a floating table. Note that, for IEEE style tables, the
% \caption command should come BEFORE the table and, given that table
% captions serve much like titles, are usually capitalized except for words
% such as a, an, and, as, at, but, by, for, in, nor, of, on, or, the, to
% and up, which are usually not capitalized unless they are the first or
% last word of the caption. Table text will default to \footnotesize as
% the IEEE normally uses this smaller font for tables.
% The \label must come after \caption as always.
%
%\begin{table}[!t]
%% increase table row spacing, adjust to taste
%\renewcommand{\arraystretch}{1.3}
% if using array.sty, it might be a good idea to tweak the value of
% \extrarowheight as needed to properly center the text within the cells
%\caption{An Example of a Table}
%\label{table_example}
%\centering
%% Some packages, such as MDW tools, offer better commands for making tables
%% than the plain LaTeX2e tabular which is used here.
%\begin{tabular}{|c||c|}
%\hline
%One & Two\\
%\hline
%Three & Four\\
%\hline
%\end{tabular}
%\end{table}


% Note that the IEEE does not put floats in the very first column
% - or typically anywhere on the first page for that matter. Also,
% in-text middle ("here") positioning is typically not used, but it
% is allowed and encouraged for Computer Society conferences (but
% not Computer Society journals). Most IEEE journals/conferences use
% top floats exclusively. 
% Note that, LaTeX2e, unlike IEEE journals/conferences, places
% footnotes above bottom floats. This can be corrected via the
% \fnbelowfloat command of the stfloats package.